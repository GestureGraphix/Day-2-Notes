\documentclass{report}

\input{preamble}
\input{macros}
\input{letterfonts}

\title{\Huge{Math 115 QR}}
\author{\huge{Alex Hernandez Juarez}}
\date{July 9 2024}

\begin{document}

\maketitle
\newpage% or \cleardoublepage
% \pdfbookmark[<level>]{<title>}{<dest>}
\pdfbookmark[section]{\contentsname}{toc}
\tableofcontents
\pagebreak

\chapter{}
\section{Warm up 1}

\qs{}{Suppose we are excavating a hole that is $10$ft long, $6$ft wide, and $4$ft deep. If the soil weighs $12$ lb/ft$^{3}$, what is the total weight of soil removed}
\sol{
	There is a uniform density 
	\begin{center}
		\[ \text{volume: } (4)(6)(10) = 240 \text{ft}^{3}\]
		\[ \text{weight: } (12 \text{lb/ft}^{3})(240 \text{ft}^{3}) = 2880 \text{lbs} \] 
	\end{center}
}

\qs{} {Suppose instead the hole is $6$ft long, $4$ft wide, and $10$ ft deep. Which hole takes more energy to dig?}

\sol{
	more energy for (2) since we need to lift the soil out of the hole
}


\section{In class notes}

\ntimg{In physics work = (force)(distance)\\
(ft * lbs) = (lbs)(ft) \\
(N * m) = (n)(M)
}

\exrose{ Q: For (2), how much work is required to remove the soil}{
	Slice the shape into layers parallel to the bottom of the hole. \\
	Let x be the depth from the top of the hole $(0 \leq x \leq 10)$ each layer has thicknes $\Delta x$. \\
	\begin{itemize}
		\item $	\text{Volume of the i}^{\text{th}} \text{ slice} = (4)(6)(\Delta x) = 24 \Delta x \text{ ft}^{3}$ 
		\item $ \text{Weight of the i}^{\text{th}} \text { slice} = 12\text{lb/ft}^{3} \space (24 \Delta x) \text{ft}^{3} = 288 \Delta x \text{lbs}$
		\item $ \text{Work needed to lift i}^{\text{th}} \approx (288 \Delta x \text{ lbs}) (x_{i} \text{ft})$ 
	\end{itemize}
	It is approximate and not exact because no all of the i$^{th}$ slice is lifted the same distance \\
	\begin{itemize}
		\item $ \text{Total work} \approx \sum^{n}_{i=1} 288x_{i} \Delta x (\text{ft * lbs})$ 
	\end{itemize}
	\begin{center}
		\[ \int^{10}_{0} 288 x dx \text{ft * lbs}  = (144x^{2}) \Big|^{10}_{0} = 14400 \text{ft * lbs} \] 

	\end{center}
}

\exrose{ Q: How much work for a $10$ by $6$ by $4$ hole}{
	\begin{itemize}
		\item Volume of the i$^{\text{th}}$ slice = $(10)(6)(\Delta x) = 60 \Delta x \text{ ft}^{3}$ 
		\item Weight of the i$^{\text{th}}$ slice = $12$lb/ft$^{3} \space (60 \Delta x) \text{ft}^{3} = 720 \Delta x$ lbs
		\item Work need to lift i$^{\text{th}} \approx (720 \Delta x \text{ lbs}) (x_{i} \text{ft})$  \\
	\end{itemize}
	It is approximate and not exact because no all of the i$^{th}$ slice is lifted the same distance \\
	\begin{center}
		\[ \text{Total work } \approx \sum^{n}_{i=1} 720x_{i} \Delta x (\text{ft * lbs}) \] 
		\[ \int^{4}_{0} 720 x dx \text{ ft * lbs}  = (360x^{2}) \Big|^{4}_{0} = 5120 \text{ft * lbs}\]
	\end{center}
}

\exrose{ Suppose the density of the soil is given by $p(x) = 8 + 2x$ lb/ft$^{3}$. How does the change the answer for the shallow soil?}{
	\begin{itemize}
		\item $\text{Volume of the i}^{th} \text{slice} = 60 \Delta x \text{ft}^{3}$
		\item $\text{Weight of the i}^{th} \text{slice} = (8 + 2x_{i})60 \Delta x $
		\item $\text{Work needed to lift the i} ^{th} \text{slice} \approx (8 + 2x+{i})(60 \Delta x)(x_{i})$
		\item $\text{Total work} \approx \sum^{n}_{i=1} (8 + 2x_{i})(60 x_{i}) \Delta $
	\end{itemize}
	\begin{center}
		\[\int^{4}_{0} (8 + 2x_{i}) (60 x) dx \] 
	\end{center}
	
}

\exrose{ Circular corn field of radius $50$ ft. The density of the corn in (ears/ft$^{2}$) is a function $f(y)$ 
where $y$ is the distance from the center of the circle. Write an intergral to compute the total yield in ears}{
	\begin{itemize}
		\item area of i$^{\text{th}}$ slice =  $\pi y_{i}^{2} - \pi(y_{i} - \Delta y)^2 = \pi y_{i}^{2} - \pi (y_{i}^{2} - 2y_{i} \Delta y + (\Delta y)^{2})$ \\
		\item ears in the i$^{th}$ slice $= p(y_{i}) 2 \pi Y_{i} \Delta y \approx 2 \pi y_{i} \Delta y$ \\
		\item total number of ears $\approx \sum^{n}_{i = 1} p(y_{i}) 2 \pi y_{i} \Delta y$ \\
		\item total number of ears $ = \int^{50}_{0} f(y) 2 \pi y \Delta y$
	\end{itemize}

}


\chapter{} 
\section{Warm up}
\qs{}{Supppose the circular corn field instead has an irrigation ditch running along a diameter. As before the density of corn is a function of the distance from the water source. How must you slice up the region so that the desnity is (approx) constant in each slice?}
\sol{
	You must slice parallel to the water source so it should be sliced vertically. 
}

\qs{}{If the density (in ears/ft$^2$) is given by the function $g$, write an integral to compute the total number of ears in the field}
\sol{ 
	\begin{center}
		\[ \left(\frac{l_{i}}{2}\right)^{2} + y_{i} = 50^{2} \]
		\[ \frac{l_{i}^2}{4} = 2500 - y_{i}^{2} \] 
		\[ l_{i} = \sqrt{10000 - 4y_{i}^2} \] 
	\end{center} 
	number of ears in $i^{th}$ slice $\approx g(y_{i}) \sqrt{10000 - 4y_{i}^2} \Delta y$ \\
	total number of ears $\approx 2 \sum^{n}_{i =1 } g(y_{i})\sqrt{10000 - 4y_{i}^2} \Delta y$ \\
	total numebr of ears $= 2 \int^{50}_{0} g(y_{i}) \sqrt{10000 - 4y_{i}^2} dy$
	}

\exrose{Conical tank full of sludge with density $f(2)$ kg/m$^{3}$, where $z$ is depth. Find an integral to compute the towrk done (against gravity) in pumping all the sludge to height $p$ 1 m aove the tank}{
	\begin{itemize}
		\item Volume of the $i^{th}$ slice $\approx \pi r_{i}^{2} \Delta x$ \\
		\item $\frac{r_{i}}{6 - z_{i}} = \frac{3}{6}$ \\
		\item $r_{i} = \frac{1}{2} ( 6 - z_{i}) = 3 - \frac{z_{i}}{2}$ \\
		\item Volume of the $i^{th}$ slice $ \pi \left( 3 - \frac{}z_{i}{2}\right)^{2} \Delta z$ m$^{3}$ \\
		\item mass of the $i^{th}$ slice $ \approx f(z_{i}) \pi \left( 3 - \frac{z_{i}}{2} \right)^{2} \Delta z$ kg \\
		\item weight of the $i^{th}$ slice $\approx 9.8f(z_{i}) \pi \left( 3 - \frac{z_{i}}{2}\right) ^{2} \Delta z$ N \\
		\item work for the $i^{th}$ slice $ \approx 9.8f(z_{i}) \pi \left( 3 - \frac{z_{i}}{2}\right) ^{2} \Delta z (z_{i} + 1)$ J\\
		\item total work $ = \sum^{n}_{i = 1} 9.8f(z_{i}) \pi \left( 3 - \frac{z_{i}}{2}\right) ^{2} (z_{i} + 1) \Delta z$ J\\
		\item total work $ =  \int_{0}^{6} 9.8f(z_{i}) \pi \left( 3 - \frac{z}{2}\right) ^{2} (z + 1) dz$ J
	\end{itemize}
}

\exrose{Given a functino f(x), $a \leq x \leq b$ what is the length of the graph}{
	\begin{itemize}
		\item length of $i^{th}$ piece $\approx \sqrt{(x_{i} - x_{i -1})^{2} + (f(x_{i}) - f(x_{i-1}))^{2}} \approx \sqrt{(\Delta x + (f'(x_{i}))\Delta x)^{2}} = \sqrt{1 + (f'(x_{i}))^{2}} \Delta x$ \\
		\item total length $= \sum^{n}_{i =1} \sqrt{1 + (f'(x_{i}))^{2}} \Delta x $ \\
		\item total length $ = \int_{a}^{b} \sqrt{1 + (f'(x_{i}))^{2}} dx$
	\end{itemize}
}

\chapter{}
\section{Warm up}
\qs{}{Find the tangent line to $f(x) = \frac{1}{1+x^{2}}$ at $x =2$}
\sol{
	\begin{center}
		\[ f'(x) = \frac{0 \cdot (1 + x^{2})  - 1(2x)}{(1 + x^{2})^{2}} = \frac{-2x}{(1+x^{2})^{2}} \]
		\[ f'(2) = \frac{-2(2)}{(1+2^{2})^{2}} = \frac{-2}{5^{2}} = \frac{-2}{25} \] 
		\[ f(2) = \frac{1}{5} \]
		\[ y - \frac{1}{5} = -\frac{2}{25}(x -2) \]  
		\[ y = \frac{1}{5} - \frac{4}{25}(x-2)\] 
	\end{center}

}

\qs{}{If we use this line to aproximate $f$ near $x=2$, will we get an overestimate or an underestimate?}
\sol{
	\begin{center}
		\[ f''(x) = \frac{-2(1+x^{2})^{2} - (-2x)2(1+x^{2})(2x)}{(1+x^{2})^{4}}\]
		\[ f''(x) = \frac{-2-2x^{2} + 8x^{2}}{(1+x^{2})^{3}}\]
		\[ f''(x) = \frac{6x^{2}-2}{(1+x^{2})^{3}} \]   
		\[ f'(x) = 0 \]
		\[ 0 = 6x^{2} -2 \]
		\[ x = \pm \sqrt{\frac{1}{3}} \]   
	\end{center}
	$L(x)$ underestimates $f(x)$ near $x=2$
}

\exrose{Find the quadratic function that best approximates $f(x) = \frac{1}{1+x^{2}}$ near $x=2$}{
	Find $Q(x) = C_{0} + C_{1}(x-2) + C_{2}(x-2)^{2}$ such that $Q(2) = f(2) = \frac{1}{5}$, $Q'(2) = f'(2) = \frac{-4}{25}$, and $G''(x) = f''(2) = \frac{22}{125}$ \\
	\begin{center}
		\[ Q'(x) = C_{1} + 2C_{2}(2-2) \] 
		\[ Q''(x) = 2c_{2} \] 
		\[ Q(2) = C_{0} = \frac{1}{5} \] 
		\[ Q'(2) = C_{1} = \frac{-4}{25}\] 
		\[ Q''(2) = 2C_{2} = \frac{22}{125} \to C_{2} = \frac{11}{125}\] 
		\[ Q(x) = \frac{1}{5} - \frac{4}{25}(x-2) + \frac{11}{125}(x-2)^{2} \] 
	\end{center}
}

\ntimg{
	General formula for Taylor Polynomial: $Q(x) = f(a) + f'(a)(x-a) + \frac{f''(x)}{2}(x-1)^{2}$ \\ 
}

\defrose{Taylor Polynomial}{The n$^{th}$ degree Taylor polynomial of $f(x)$ based at $x=a$ is: \\
	$P_{n}(x) = f(a) + f'(a)(x-a) + \frac{f''(a)}{2}(x-a)^{2} +  \frac{f^{(3)}(a)}{6}(x-a)^{3} + \frac{f^{(4)}(a)}{24}(x-a)^{4} + \ldots \frac{f^{(n)}(a)}{n!}(x-a)^{n}$ \\
	\begin{center}
		\[ \sum^{n}_{k=0} \frac{f^{(k)}(a)}{k!}(x-a)^{k} \] 
	\end{center}
	Define: \\
	\begin{itemize}
		\item 	$f^{(0)}(x) = f(x)$ 
		\item 	$0! =1$
		\item 	$(x-a)^{0} = 1$ for all $x$ 
	\end{itemize}
}

\exrose{Let $f(x) = \frac{1}{1-x}$, based at $x=0$. Find $P_{4}(x)$}{
	\begin{center}
		\[ f(x) = \frac{1}{1-x}\]
		\[ f'(x) = \frac{-1}{(1-x)^{2}}(-1) = \frac{1}{(1-x)^{2}}\]  
		\[ f''(x) = \frac{2}{(1-x)^{3}}\] 
		\[ f^{(3)}(x) = \frac{6}{(1-x)^{4}} \]
		\[ f^{(4)} = \frac{24}{(1-x)^{5}}\]   
	\end{center}
}

\chapter{}
\section{Warm Up}
\qs{}{Suppose for a certain function $f$ we know that $|f'(x)| \leq f$ for all x. \\
(1) What is the largest possible value of $\left| f(4) - f(1) \right|$? \\\
(2) What is the largest possible value of $|f(b) - f(a)|$ for a given interval $[a, b]$? \\
(3) Suppose we also know that f(1) = 10. Find upper and lower bound for $f(4)$. }
\sol{
	\\
	(1) \\
	$15$ \\
	(2) \\
	$(b-a)5$\\
	(3) \\
	Upper: $25$ \\
	Lower: $-5$ \\
}

\ntimg{
	Mean value theorem: If $f$ is  continuous on $[a,b]$ and differentiable on the $(a,b)$ then there 
	exists a $c$ on $(a,b)$ such that $f'(c) = \frac{f(b) - f(a)}{b-a}$ \\\\
	Taylor's Theorem: If $f$ is continuous on $[a,b]$ and $(n+1)^{\text{st}}$ differentiable 
	on $(a,b)$ then there exists a $c$ in $(a,b)$ such that $f(b) = f(a) + f'(a)(b-a) + \frac{f'(a)}{2}(b-a) + \dots + \frac{f^{(n)}(a)}{n!} (b-a)^{n} + \frac{f^{(n+1)}(c)}{(n+1)!}(b-a)^{n+1}$ \\\\
	Mean Value Theorem as the Taylor's Theorem at $n = 0$. \\
	\begin{center}
		\[f(b) = f(a) + \frac{f'(c)}{1!}(b-a)^{1}\]
		\[ f(b) - f(a) = f'(c)(b-a)\]
		\[ f'(c) = \frac{f(b) - f(a)}{b - a}\] 
	\end{center}

	$P_{n}(b)$: Where $P_{n} (x)$ is $n^{th}$ degree Taylor Poly for $f(x)$ based at $x = a$. So, 
	\begin{center}
		\[ |f(b) - P_{n}(b)| = \left|\frac{f^{(n + 1)}(c)}{(n + 1)!}(b-a)^{n+1}\right| = \frac{\left| f^{(n+1)}(c)\right|(b-a)^{n+1}}{(n+1)!} \] 
	\end{center} 
	Error Bound: $|f(x) - P_{n}(x)| \leq \frac{M}{(n+1)!}(x-a)^{n+1}$ \\
	$n^{th}$ degree TP based at x = a: where $|f^{(n+1)} (z)| \leq M$ for all $z$ between $a$ and $x$. 
}

\exrose{In quiz, we wanted to approximate $\sqrt{3.95}$ using 1$^{st}$ degree TP based at $x=4$.}{
	\begin{center}
		\[ P_{1}(x) = 2 + \frac{1}{4}(x-a) \]
		\[ \sqrt{3.95} \approx P_{1}(3.95) = 2 + \frac{1}{4} (3.95 - 4) = 2 - 0.00125 = 1.9875\] 
		\[ f(x) = x^{\frac{1}{2}}\]
		\[ f'(x) = \frac{1}{2} x ^{-\frac{1}{2}}\]
		\[ |f''(x) | = \left| \frac{1}{4}x^{-\frac{3}{1}} \right| = \frac{1}{4} x^{-\frac{3}{2}} \]   
		\[ M = \frac{1}{4}: \left| \sqrt{3.95} - 1.9875 \right| \leq \frac{\frac{1}{4} \left| 3.95 - 4\right| ^{2}}{2} \approx 0.00004\] 
	\end{center}
}

\exrose{Find $n$ so that the $n^{th}$ defree TP for $f(x) = \cos x $ based at $x = 0$ approximate $\cos(0.03)$ to within $10^{-15}$}{
	\begin{center}
		\[ f'(x) = - \sin(x) \]
		\[ f''(x) = - \cos (x) \]
		\[ f^{(3)} (x) = \sin(x) \] 
		\[ f^{(4)} (x) = \cos(x) \]   
		\[ \left| f^{(n+1)} \leq 1\right| \text{ eveywhere and for every n}\]
		\[ M = 1\] 
		\[ \cos(0.03) - P_{n}(0.03) \leq \frac{1 \cdot (0.03 - 0)^{n+1}}{n+1} = \left(\frac{3}{1000}\right)^{n+1} \frac{1}{(n+1)!} \leq 10^{-15} \]  
	\end{center}
}

\chapter{}
\section{}{Warm up}
\qs{}{(a) Find the $5^{th}$ degree Taylor Polynomial for $f(x) = e^{x}$ based at $x = 0$.\\
(b) Find $P_{n}(x)$. \\
(c) Find an upper bound (in terms of n ) of the error $\left|f(x) - P_{n}(x)\right|$, for $x > 0$\\
(d) What is $\lim_{n \to \infty} |f(x) - P_{n}(x)$}
\sol{
	(a)
	\begin{center}
		\[ f(x) = e^{x}\] 
		\[ f'(x) = e^{x}\] 
		\[ f''(x) = e^{x}\] 
		\[ f^{(2)}(x) = e^{x}\] 
		\[ f^{(3)}(x) = e^{x}\] 
		\[ f^{(4)}(x) = e^{x}\] 
		\[ f^{(5)}(x) = e^{x}\] 
		\[ P_{5}(x) = e^{0} + e^{0}(x) + \frac{e^{0}}{2}(x)^{2} + \frac{e^{0}}{6}(x)^{3} + \frac{e^{0}}{24}(x)^{4} + \frac{e^{0}}{120}(x)^{5}\] 
		\[ P_{5}(x) = 1 + (x) + \frac{1}{2}(x)^{2} + \frac{1}{6}(x)^{3} + \frac{1}{24}(x)^{4} + \frac{1}{120}(x)^{5}\] 
	\end{center}
	(b) 
	\begin{center}
		\[ P_{n}(x) = \sum^{n}_{k=0} \frac{1}{k!}(x)^{k}\] 
	\end{center}
	(c) 
	\begin{center}
		\[ |f(x) - P_{n}(x)| \leq \frac{M(x-a)^{n+1}}{(n+1)}\] 
		\[ |f(x) - P_{n}(x)| \leq \frac{e^{x}(x)^{n+1}}{(n+1)!}, \text{where } |e^{z}| \leq M \text{ for all } z \text{ between 0 and } x \] 
	\end{center}
	(d) 
	\begin{center}
		\[ \lim_{n \to \infty} |f(x) - P_{n}(x) = ? \]
		\[ \lim_{n \to \infty } \frac{e^{x} \cdot x^{n+1}}{(n+1)!}\]  
		\[ e^{x} \lim_{n \to \infty } \frac{x^{n+1}}{(n+1)!} = 0 \] 
	\end{center}

\section{Infinite Series}
\exrose{Consider $x = 1$ in the warm up. We showed $\lim_{n \to \infty} |e^{1} - P_{n}(1)| = 0 $. ie $\lim_{n \to \infty} | e - \left( 1 +  1 \frac{1}{2}   + \frac{1}{6}  + \frac{1}{24} + \ldots + \frac{1}{n!}\right)| = 0$ \\ $\lim_{n \to \infty}  \left( 1 + 1  + \frac{1}{2}  + \frac{1}{6}  + \frac{1}{24} + \ldots +  + \frac{1}{n!}\right) = e$}{
	\text{i.e. $\sum_{k = 0}^{\infty} \frac{1}{k!} = e$} }
}

\ntimg{
	Def: An infinite series is a sum of the form 
	\begin{center}
		\[\sum_{k = 0 }^{\infty} a_{k} = a_{0} + a_{1} + a_{2} + \ldots \]
	\end{center}
	Def: Given an infinite series $ \sum_{k = 0 }^{\infty} a_{k}$  the $n^{th}$ partial sum is 
	\begin{center}
		\[ S_{n} = \sum_{k =0 }^{n} a_{k}\] 
	\end{center} 
}

\exrose{Example 1}{
	\begin{center}
		\[ \sum_{k=0}^{\infty} \frac{1}{k!} = 1 + 1 + \frac{1}{2} + \frac{1}{6} + \ldots \]
		\[ S_{0} = \sum_{k = 0}^{0} \frac{1}{k!} = \frac{1}{0!} = 1\]
		\[ S_{1} = \sum_{k = 0}^{1} \frac{1}{k!} = \frac{1}{0!} + \frac{1}{1!} = 2\]  
		\[ S_{2} = \sum_{k = 0}^{2} \frac{1}{k!} = \frac{1}{0!} + \frac{1}{1!} + \frac{1}{2!}= \frac{5}{2}\]  
		\[ S_{3} = \sum_{k = 0}^{3} \frac{1}{k!} = \frac{1}{0!} + \frac{1}{1!} + \frac{1}{2!} + \frac{1}{3!}= \frac{8}{3}\]  
	\end{center}
}

\exrose{Example 2}{
	\begin{center}
		\[ \sum_{k = 0}^{\infty} (-1)^{k} = 1 - 1 + 1 - 1 + 1 \dots \]
		\[ S_{0} = 1\]  
		\[ S_{1} = 1 - 1 = 0\]  
		\[ S_{2} = S_{1} + 1 = 1\] 
		\[ S_{3} = S_{2} - 1 = 0\]  

	\end{center}
}

\exrose{Example 3}{
	\begin{center}
		\[ \sum_{k = 2}^{\infty} \frac{1}{k^{2} + k}\]
		\[ S_{2} = \frac{1}{6} \] 
		\[ S_{3} = S_{2} + \frac{1}{3^{2} + 3} = \frac{1}{6} + \frac{1}{12} = \frac{1}{4}\]
		\[ S_{4} = S_{3} + \frac{1}{16 + 4} = \frac{1}{12} + \frac{1}{20} = \frac{3}{10}\]   
	\end{center}
}

\ntimg{Def: $\sum_{k = 0}^{\infty} a_{k} $ converges if $\lim_{n \to \infty} S_{n}$ exists and is finite, in which case
we write 
	\begin{center}
		\[ \sum_{k = 0}^{\infty} a_{k} = \lim_{n \to \infty} S_{n}\]
	\end{center}
	if $\lim_{n \to \infty } S_{n}$ does not exist (including $\pm \infty$) the series diverges
	
	Ex(1) 
	\begin{center}
		\[ \sum_{k =0}^{\infty} \frac{1}{k!} \text{ converges because } \lim_{n \to \infty } \sum_{k =0}^{\infty} \frac{1}{k!} = e\]
	\end{center}
	Ex(2) 
	\begin{center}
		
	\end{center}
}

\chapter{}
\section{}{Warm up}
\qs{}{(a)Verify using algebra that $\frac{1}{k^{2} + k}$ \\
(b) Because of (1), we can write the seris as 		$R S_{n} = \left(\frac{1}{2} - \frac{1}{3}\right) + \left(\frac{1}{3} - \frac{1}{3}\right) + \left(\frac{1}{4} - \frac{1}{5}\right) + \ldots \left(\frac{1}{n} - \frac{1}{n+1}\right)$ use this to form a close formula
for the n$^{th}$ partial sum. \\
(c) Does the series converge?}
\sol{
	\begin{center}
		\[ \frac{1}{k^{2} + k}\] 
		\[ \frac{1}{k(k+1)}\] 
		\[ \frac{A}{k} + \frac{B}{k+1} = \frac{1}{k(k+1)}\]
		\[ \frac{A(k+1) + B(k)}{k^{2}+k}\]
		\[ A(-1 + 1) + B(-1) = -B = 1\]   
		\[ A(1 + 0) + B(0) = A = 1\]   

	\end{center}
	(b)
	\begin{center}
		\[ S_{n} = \left(\frac{1}{2} - \frac{1}{3}\right) + \left(\frac{1}{3} - \frac{1}{3}\right) + \left(\frac{1}{4} - \frac{1}{5}\right) + \ldots \left(\frac{1}{n} - \frac{1}{n+1}\right)\] 
		\[ S_{n} = \frac{1}{2} - \frac{1}{n+1}\] 
	\end{center}
	(c) 
	Yes to $\frac{1}{2}$
}

\section{Geometric Series}
\exrose{Suppose a patient takes 100mg of a certain druge once per day. In 
24 hours the patients body eliminates 80$\text{\%}$ of the drug}{
	(a) If the first dose is the $k = 0$ dose, how much of the drug is present 
	immediately after the $ k = 1$ dose 
	\begin{center}
		\[ S_{1} = 100 + 100(0.2) \] 
	\end{center}
	(b) $k = 2$?
	\begin{center}
		\[ S_{2} = 100 + 120(0.2) = 100 + 100(0.2) + 100(0.2)^{2} = 124\] 
	\end{center}
	(c) $ = 3$
	\begin{center}
		\[ S_{3} = 100 + 124(0.2) =  100 + 100(0.2) + 100(0.2)^{2} + 100(0.2)^{3}  100 + 100(0.2) + 100(0.2)^{2}= 124.8\] 
		\[ S_{n} = 100 + 124(0.2) =  100 + 100(0.2) + 100(0.2)^{2} + 100(0.2)^{3}  100 + 100(0.2) + 100(0.2)^{2} + \ldots + 100(0.2)^{n}\] 
		\[ S_{n} = \sum_{k=0}^{n} 100(0.2)^{k}\] 
	\end{center}
	(d) What happens in the long term?
	\begin{center}
		\[ \sum_{k=0}^{\infty} 100(0.2)^{k} = \frac{100}{1-0.2} = \frac{100}{0.8} = 125 mg \] 
	\end{center}

}

\ntimg{In general a geometric series has the form
	\begin{center}
		\[ \sum_{k = 0}^{\infty} a r ^{k} = a + ar + ar^{2}\] 
	\end{center}
	a is the first term. 
	\\ r is the common ratio between succesive terms. \
	\begin{center}
		\[ S_{n} = a + ar + ar^{2} + \ldots + ar^{n} \] 
		\[ rS_{n} = ar + ar ^{2} + ar^{3} + \ldots + ar^{n+1}\]
		\[ (1-r)S_{n} = a\left(1-r^{n+1}\right)\] 
		\[ S_{n} = a\left(\frac{1-r^{n+1}}{1 - r}\right)\] 
	\end{center}
	$r^{n+1} \to 0$ if $|r| < 1$ \\
	$r^{n+1} \to \infty$ if $|r| > 1$ \\
	If $|r| < 1$, the series $\sum_{k=0}^{\infty} ar^{k}$ converges to $\frac{a}{1-r}$. 
	Otherwise, the series diverges
}

\section{Iv}

\nt{Harmonic Series \\
	\begin{center}
		\[ \sum_{k =1}^{\infty} \frac{1}{k} = \frac{1}{2} + \frac{1}{4} + \frac{1}{4} + \frac{1}{5} + \ldots \] 
	\end{center}
}

\chapter{}
\section{Warm Up}

\qs{}{Which of the following statements must be true about a series and its partial sums, which 
	must be false, and for which is it impossible to say \\
	(1) If $a_{k} > 0$ for all $k$, then $S_{n} = \sum_{k =1}^{n} a_{k}$ is increasing \\
	(2) If $\lim_{n \to \infty } a_{k} = 0$, then $lim_{n \to \infty} s_{n}$ exists and is finite \\
	(3) If $S_{n} = \sum_{k=1}^{n} a_{k} = \frac{4n^{2}}{1+n^{2}}$, then $a_{2} = \frac{6}{5}$ \\
	(4) If $S_{n} = \sum_{k = 1}^{n} a_{k} = \frac{4n^{2}}{1 + n^{2}}$ then the series converges.
	}

\sol{
	\\
	(1) true $s_{n+1} = s_{n} + a_{n+1} > s_{n}$ since $a_{n}>0$\\
	(2) its $\sum_{k=1}^{n} \frac{1}{k}$ diverges and $\sum_{k=1}^{n} = \frac{1}{k^{2}}$ converges\\
	(3) 
	\begin{center}
		\[ S_{1} = \frac{4(1)^{2}}{1+1^{2}} = \frac{4}{2} = 2\] 
		\[ a_{2} = S_{2} - S_{1} = \frac{4(2)^{2}}{1 + (2)^{2}} - 2 = \frac{16}{5} - 2\] 
	\end{center}
	true.\\
	(4) true, $\lim_{n \to \infty} \frac{4n^{2}}{1 + n^{2}} = 4$
}

\exrose{Cconsider a bowl whose shpae is given by rotating the grahp of $y =x^{2}$, $o \leq x \leq 2$ around the 
y-axis}{
	(a) Write an integral to compute the volume of the bowl. 
	\begin{center}
		\[ \text{volume of the i}^{th} \text{ slice } \pi r_{i}^{2} \Delta h \] 
		\[ h_{i} = r_{i}^{2} \] 
		\[ r_{i} = \sqrt{h_{i}} \] 
		\[ \pi \left(\sqrt{h_{i}}\right)^{2} \Delta h\]
		\[ \text{ total volume } \approx \sum_{k=1}^{n} \pi h_{i} \Delta h\]
		\[ \text{ total volume} = \int_{0}^{4} \pi h dh \text{ cm}^{3}\] 

	\end{center}
	(b) If density of oatmeal bowm $\rho (h)$ g/cm$^{3}$  where $h$ is the veritcal distance from the bottom, 
	$0 \leq h \leq 4$ write an integral to compute the mass of oatmeal in the bowl.
	\begin{center}
		\[ \text{ mass of i}^{th} \text{ slice} = \rho (h_{i}) \pi h_{i} \Delta h\]
		\[ \text{ total mass} = \int_{0}^{4} \rho(h) \pi h dh\]  
	\end{center} 
	(c) Write an integral to compute work required to lift all oatmeal to height of 15 cm above top of bowl. 
	\begin{center}
		\[ g = \frac{\rho(h_{i})\pi \Delta h}{1000}\] 
		\[ \text{ work for i}^{th}\text{ slice } = \frac{9.8(\rho(h_{i}))\pi h_{i} \Delta h}{1000} \left(\frac{19 - h_{i}}{100}\right)\]
		\[ \text{total work} = \int_{0}^{4} \frac{9.8\rho(h)\pi h (19 - h)}{100000} dh\] 
	\end{center}
}

\ntimg{ 
	If $f$ is increasing, $L_{n} < I < R_{n}$ \\
	If $f$ is decreasing, $R_{n} < I < L_{n}$ \\
	If $f$ is concave up, $M_{n} < I < T_{n}$ \\
	If $f$ is concave down, $T_{n} < I < M{n}$
}
\exrose{Let $I = \int_{0}^{8} \frac{1}{x+3} dx$}{
	(a) Is $M_{2}$ an over or under approx. for I? \\
	\begin{center}
		\[ f(x) = \frac{1}{x + 3} = (x + 8)^{-1}\]
		\[ f'(x) = -(x + 3)^{-2}\]
		\[ f''(x) = 2(x+3)^{-3} \]   
	\end{center}
	$2(x+3)^{-3} > 0 $ on $[0,8]$. underestimate. 
	(b) Compute $M_{2}$ \\ 
	\begin{center}
		\[ M_{2} = f(2) \Delta x + f(6) \Delta x\]
		\[ M_{2} = \frac{1}{5}(4)  +\frac{1}{9}(4) \] 
		\[ M_{2} = \frac{56}{45}\]  
	\end{center}
	(c) Find an upper bound on the error when using $M_{2}$ to approximate I. 
	\begin{center}
		\[ |I - M_{n} \leq \frac{M(b-a)^{3}}{24n^{2}}, \text{ where } |f''(x)| \leq M_{n} \text{ on } [0,8]\]
		\[ f''(x) = \frac{2}{(x+3)^{3}} \leq \frac{2}{27} \text{ on } [0,8] \]
		\[ |I - M_{n}| \leq \frac{\frac{2}{27}(8 - 0)}{24(2)^{2}} = \frac{2^{5}}{3^{4}} = \frac{32}{81}\]    
	\end{center} 
}

\end{document}
