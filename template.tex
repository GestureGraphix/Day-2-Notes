\documentclass{report}

\input{preamble}
\input{macros}
\input{letterfonts}

\title{\Huge{Math 115 QR}}
\author{\huge{Alex Hernandez Juarez}}
\date{July 9 2024}

\begin{document}

\maketitle
\newpage% or \cleardoublepage
% \pdfbookmark[<level>]{<title>}{<dest>}
\pdfbookmark[section]{\contentsname}{toc}
\tableofcontents
\pagebreak

\chapter{}
\section{Warm up 1}

\qs{}{Suppose we are excavating a hole that is $10$ft long, $6$ft wide, and $4$ft deep. If the soil weighs $12$ lb/ft$^{3}$, what is the total weight of soil removed}
\sol{
	There is a uniform density 
	\begin{center}
		\[ \text{volume: } (4)(6)(10) = 240 \text{ft}^{3}\]
		\[ \text{weight: } (12 \text{lb/ft}^{3})(240 \text{ft}^{3}) = 2880 \text{lbs} \] 
	\end{center}
}

\qs{} {Suppose instead the hole is $6$ft long, $4$ft wide, and $10$ ft deep. Which hole takes more energy to dig?}

\sol{
	more energy for (2) since we need to lift the soil out of the hole
}

\section{In class notes}

\nt{In physics work = (force)(distance)\\
(ft * lbs) = (lbs)(ft) \\
(N * m) = (n)(M)
}

\ex{ Q: For (2), how much work is required to remove the soil}{
	Slice the shape into layers parallel to the bottom of the hole. \\
	Let x be the depth from the top of the hole $(0 \leq x \leq 10)$ each layer has thicknes $\Delta x$. \\
	Volume of the i$^{\text{th}}$ slice = $(4)(6)(\Delta x) = 24 \Delta x \text{ ft}^{3}$ \\
	Weight of the i$^{\text{th}}$ slice = $12$lb/ft$^{3} \space (24 \Delta x) \text{ft}^{3} = 288 \Delta x$ lbs \\
	Work need to lift i$^{\text{th}} \approx (288 \Delta x \text{ lbs}) (x_{i} \text{ft})$  \\
	It is approximate and not exact because no all of the i$^{th}$ slice is lifted the same distance \\
	Total work $\approx \sum^{n}_{i=1} 288x_{i} \Delta x (\text{ft * lbs})$ \\
	$\int^{10}_{0} 288 x dx $ ft * lbs $ = (144x^{2}) \Big|^{10}_{0} = 14400$ ft * lbs 
}

\ex{ Q: How much work for a $10$ by $6$ by $4$ hole}{
	Volume of the i$^{\text{th}}$ slice = $(10)(6)(\Delta x) = 60 \Delta x \text{ ft}^{3}$ \\
	Weight of the i$^{\text{th}}$ slice = $12$lb/ft$^{3} \space (60 \Delta x) \text{ft}^{3} = 720 \Delta x$ lbs \\
	Work need to lift i$^{\text{th}} \approx (720 \Delta x \text{ lbs}) (x_{i} \text{ft})$  \\
	It is approximate and not exact because no all of the i$^{th}$ slice is lifted the same distance \\
	Total work $\approx \sum^{n}_{i=1} 720x_{i} \Delta x (\text{ft * lbs})$ \\
	$\int^{4}_{0} 720 x dx $ ft * lbs $ = (360x^{2}) \Big|^{4}_{0} = 5120$ ft * lbs 
}

\ex{ Suppose the density of the soil is given by $p(x) = 8 + 2x$ lb/ft$^{3}$. How does the change the answer for the shallow soil?}{
	Volume of the i$^{th}$ slice $= 60 \Delta x \text{ft}^{3}$ \\
	Weight of the i$^{th}$ slice $= (8 + 2x_{i})60 \Delta x$ \\
	Work needed to lift the i$^{th}$ slice $\approx (8 + 2x+{i})(60 \Delta x)(x_{i})$ \\
	Total work $ \approx \sum^{n}_{i=1} (8 + 2x_{i})(60 x_{i}) \Delta x$ \\
	$\int^{4}_{0} (8 + 2x_{i}) (60 x) dx$ 
}

\ex{ Circular corn field of radius $50$ ft. The density of the corn in (ears/ft$^{2}$) is a function $f(y)$ 
where $y$ is the distance from the center of the circle. Write an intergral to compute the total yield in ears}{
	area of i$^{\text{th}}$ slice =  $\pi y_{i}^{2} - \pi(y_{i} - \Delta y)^2$ \\
	= $\pi y_{i}^{2} - \pi (y_{i}^{2} - 2y_{i} \Delta y + (\Delta y)^{2})$ \\
	$ \approx 2 \pi y_{i} \Delta y$ \\
	ears in the i$^{th}$ slice $= p(y_{i}) 2 \pi Y_{i} \Delta y$ \\
	total number of ears $\approx \sum^{n}_{i = 1} p(y_{i}) 2 \pi y_{i} \Delta y$ \\
	total number of ears $ = \int^{50}_{0} f(y) 2 \pi y \Delta y$

}


\chapter{} 
\section{Warm up}
\qs{}{Supppose the circular corn field instead has an irrigation ditch running along a diameter. As before the density of corn is a function of the distance from the water source. How must you slice up the region so that the desnity is (approx) constant in each slice?}
\sol{
	You must slice parallel to the water source so it should be sliced vertically. 
}

\qs{}{If the density (in ears/ft$^2$) is given by the function $g$, write an integral to compute the total number of ears in the field}
\sol{ 
	\begin{center}
		\[ \left(\frac{l_{i}}{2}\right)^{2} + y_{i} = 50^{2} \]
		\[ \frac{l_{i}^2}{4} = 2500 - y_{i}^{2} \] 
		\[ l_{i} = \sqrt{10000 - 4y_{i}^2} \] 
	\end{center} 
	number of ears in $i^{th}$ slice $\approx g(y_{i}) \sqrt{10000 - 4y_{i}^2} \Delta y$ \\
	total number of ears $\approx 2 \sum^{n}_{i =1 } g(y_{i})\sqrt{10000 - 4y_{i}^2} \Delta y$ \\
	total numebr of ears $= 2 \int^{50}_{0} g(y_{i}) \sqrt{10000 - 4y_{i}^2} dy$
	}

\ex{Conical tank full of sludge with density $f(2)$ kg/m$^{3}$, where $z$ is depth. Find an integral to compute the towrk done (against gravity) in pumping all the sludge to height $p$ 1 m aove the tank}{
	Volume of the $i^{th}$ slice $\approx \pi r_{i}^{2} \Delta x$ \\
	$\frac{r_{i}}{6 - z_{i}} = \frac{3}{6}$ \\
	$r_{i} = \frac{1}{2} ( 6 - z_{i}) = 3 - \frac{z_{i}}{2}$ \\
	Volume of the $i^{th}$ slice $ \pi \left( 3 - \frac{}z_{i}{2}\right)^{2} \Delta z$ m$^{3}$ \\
	mass of the $i^{th}$ slice $ \approx f(z_{i}) \pi \left( 3 - \frac{z_{i}}{2} \right)^{2} \Delta z$ kg \\
	weight of the $i^{th}$ slice $\approx 9.8f(z_{i}) \pi \left( 3 - \frac{z_{i}}{2}\right) ^{2} \Delta z$ N \\
	work for the $i^{th}$ slice $ \approx 9.8f(z_{i}) \pi \left( 3 - \frac{z_{i}}{2}\right) ^{2} \Delta z (z_{i} + 1)$ J\\
	total work $ = \sum^{n}_{i = 1} 9.8f(z_{i}) \pi \left( 3 - \frac{z_{i}}{2}\right) ^{2} (z_{i} + 1) \Delta z$ J\\
	total work $ =  \int_{0}^{6} 9.8f(z_{i}) \pi \left( 3 - \frac{z}{2}\right) ^{2} (z + 1) dz$ J
}

\ex{Given a functino f(x), $a \leq x \leq b$ what is the length of the graph}{
	length of $i^{th}$ piece $\approx \sqrt{(x_{i} - x_{i -1})^{2} + (f(x_{i}) - f(x_{i-1}))^{2}} \approx \sqrt{(\Delta x + (f'(x_{i}))\Delta x)^{2}} = \sqrt{1 + (f'(x_{i}))^{2}} \Delta x$ \\
	total length $= \sum^{n}_{i =1} \sqrt{1 + (f'(x_{i}))^{2}} \Delta x $ \\
	total length $ = \int_{a}^{b} \sqrt{1 + (f'(x_{i}))^{2}} dx$
}

\chapter{}
\section{Warm up}
\qs{}{Find the tangent line to $f(x) = \frac{1}{1+x^{2}}$ at $x =2$}
\sol{
	\begin{center}
		\[ f'(x) = \frac{0 \cdot (1 + x^{2})  - 1(2x)}{(1 + x^{2})^{2}} = \frac{-2x}{(1+x^{2})^{2}} \]
		\[ f'(2) = \frac{-2(2)}{(1+2^{2})^{2}} = \frac{-2}{5^{2}} = \frac{-2}{25} \] 
		\[ f(2) = \frac{1}{5} \]
		\[ y - \frac{1}{5} = -\frac{2}{25}(x -2) \]  
		\[ y = \frac{1}{5} - \frac{4}{25}(x-2)\] 
	\end{center}

}

\qs{}{If we use this line to aproximate $f$ near $x=2$, will we get an overestimate or an underestimate?}
\sol{
	\begin{center}
		\[ f''(x) = \frac{-2(1+x^{2})^{2} - (-2x)2(1+x^{2})(2x)}{(1+x^{2})^{4}}\]
		\[ f''(x) = \frac{-2-2x^{2} + 8x^{2}}{(1+x^{2})^{3}}\]
		\[ f''(x) = \frac{6x^{2}-2}{(1+x^{2})^{3}} \]   
		\[ f'(x) = 0 \]
		\[ 0 = 6x^{2} -2 \]
		\[ x = \pm \sqrt{\frac{1}{3}} \]   
	\end{center}
	$L(x)$ underestimates $f(x)$ near $x=2$
}

\ex{Find the quadratic function that best approximates $f(x) = \frac{1}{1+x^{2}}$ near $x=2$}{
	Find $Q(x) = C_{0} + C_{1}(x-2) + C_{2}(x-2)^{2}$ such that $Q(2) = f(2) = \frac{1}{5}$, $Q'(2) = f'(2) = \frac{-4}{25}$, and $G''(x) = f''(2) = \frac{22}{125}$ \\
	\begin{center}
		\[ Q'(x) = C_{1} + 2C_{2}(2-2) \] 
		\[ Q''(x) = 2c_{2} \] 
		\[ Q(2) = C_{0} = \frac{1}{5} \] 
		\[ Q'(2) = C_{1} = \frac{-4}{25}\] 
		\[ Q''(2) = 2C_{2} = \frac{22}{125} \to C_{2} = \frac{11}{125}\] 
		\[ Q(x) = \frac{1}{5} - \frac{4}{25}(x-2) + \frac{11}{125}(x-2)^{2} \] 
	\end{center}

}

\nt{
	General formula: $Q(x) = f(a) + f'(a)(x-a) + \frac{f''(x)}{2}(x-1)^{2}$ \\
	Def: The n$^{th}$ degree Taylor polynomial of $f(x)$ based at $x=a$ is: \\
	$P_{n}(x) = f(a) + f'(a)(x-1) + \frac{f''(a)}{2}(x-a)^{2} +  \frac{f^{(3)}(a)}{6}(x-a)^{3} + \frac{f^{(4)}(a)}{24}(x-a)^{4} + \ldots \frac{f^{(n)}(a)}{n!}(x-a)^{n}$ \\
	$\sum^{n}_{k=0} \frac{f^{(k)}(a)}{k!}(x-a)^{k}$
	Define: \\
	$f^{(0)}(x) = f(x)$
	$0! =1$\\
	$(x-a)^{0} = 1$ for all $x$  
}

\ex{Let $f(x) = \frac{1}{1-x}$, based at $x=0$. Find $P_{4}(x)$}{
	\begin{center}
		\[ f(x) = \frac{1}{1-x}\]
		\[ f'(x) = \frac{-1}{(1-x)^{2}}(-1) = \frac{1}{(1-x)^{2}}\]  
		\[ f''(x) = \frac{2}{(1-x)^{3}}\] 
		\[ f^{(3)}(x) = \frac{6}{(1-x)^{4}} \]
		\[ f^{(4)} = \frac{24}{(1-x)^{5}}\]   
	\end{center}
}

\chapter{}
\section{Warm Up}
\qs{}{Suppose for a cerntain function $f$ we know that $|f'(x)| \leq f$ for all x. \\
(1) What is the largest possible value of $\left| f(4) - f(1) \right|$? \\\
(2) What is the largest possible value of $|f(b) - f(a)|$ for a given interval $[a, b]$? \\
(3) Suppose we also know that f(1) = 10. Find upper and lower bound for $f(4)$. }
\sol{
	\\
	(1) \\
	$15$ \\
	(2) \\
	$(b-a)5$\\
	(3) \\
	Upper: $25$ \\
	Lower: $-5$ \\
}

\nt{
	Mean value theorem: If $f$ is  continuous on $[a,b]$ and differentiable on the $(a,b)$ then there 
	exists a $c$ on $(a,b)$ such that $f'(c) = \frac{f(b) - f(a)}{b-a}$ \\\\
	Taylor's Theorem: If $f$ is continuous on $[a,b]$ and $(n+1)^{\text{st}}$ differentiable 
	on $(a,b)$ then there exists a $c$ in $(a,b)$ such that $f(b) = f(a) + f'(a)(b-a) + \frac{f'(a)}{2}(b-a) + \dots + \frac{f^{(n)}(a)}{n!} (b-a)^{n} + \frac{f^{(n+1)}(c)}{(n+1)!}(b-a)^{n+1}$ \\\\
	Mean Value Theorem as the Taylor's Theorem at $n = 0$. \\
	\begin{center}
		\[f(b) = f(a) + \frac{f'(c)}{1!}(b-a)^{1}\]
		\[ f(b) - f(a) = f'(c)(b-a)\]
		\[ f'(c) = \frac{f(b) - f(a)}{b - a}\] 
	\end{center}

	$P_{n}(b)$: Where $P_{n} (x)$ is $n^{th}$ degree Taylor Poly for $f(x)$ based at $x = a$. So, 
	\begin{center}
		\[ |f(b) - P_{n}(b)| = \left|\frac{f^{(n + 1)}(c)}{(n + 1)!}(b-a)^{n+1}\right| = \frac{\left| f^{(n+1)}(c)\right|(b-a)^{n+1}}{(n+1)!} \] 
	\end{center} 
	Error Bound: $|f(x) - P_{n}(x)| \leq \frac{M}{(n+1)!}(x-a)^{n+1}$ \\
	$n^{th}$ degree TP based at x = a: where $|f^{(n+1)} (z)| \leq M$ for all $z$ between $a$ and $x$. 
}

\ex{In quiz, we wanted to approximate $\sqrt{3.95}$ using 1$^{st}$ degree TP based at $x=4$.}{
	\begin{center}
		\[ P_{1}(x) = 2 + \frac{1}{4}(x-a) \]
		\[ \sqrt{3.95} \approx P_{1}(3.95) = 2 + \frac{1}{4} (3.95 - 4) = 2 - 0.00125 = 1.9875\] 
		\[ f(x) = x^{\frac{1}{2}}\]
		\[ f'(x) = \frac{1}{2} x ^{-\frac{1}{2}}\]
		\[ |f''(x) | = \left| \frac{1}{4}x^{-\frac{3}{1}} \right| = \frac{1}{4} x^{-\frac{3}{2}} \]   
		\[ M = \frac{1}{4}: \left| \sqrt{3.95} - 1.9875 \right| \leq \frac{\frac{1}{4} \left| 3.95 - 4\right| ^{2}}{2} \approx 0.00004\] 
	\end{center}
}

\ex{Find $n$ so that the $n^{th}$ defree TP for $f(x) = \cos x $ based at $x = 0$ approximate $\cos(0.03)$ to within $10^{-15}$}{
	\begin{center}
		\[ f'(x) = - \sin(x) \]
		\[ f''(x) = - \cos (x) \]
		\[ f^{(3)} (x) = \sin(x) \] 
		\[ f^{(4)} (x) = \cos(x) \]   
		\[ \left| f^{(n+1)} \leq 1\right| \text{ eveywhere and for every n}\]
		\[ M = 1\] 
		\[ \cos(0.03) - P_{n}(0.03) \leq \frac{1 \cdot (0.03 - 0)^{n+1}}{n+1} = \left(\frac{3}{1000}\right)^{n+1} \frac{1}{(n+1)!} \leq 10^{-15} \]  
	\end{center}
}

\chapter{}
\section{}{Warm up}
\qs{}{(a) Find the $5^{th}$ degree Taylor Polynomial for $f(x) = e^{x}$ based at $x = 0$.\\
(b) Find $P_{n}(x)$. \\
(c) Find an upper bound (in terms of n ) of the error $\left|f(x) - P_{n}(x)\right|$, for $x > 0$\\
(d) What is $\lim_{n \to \infty} |f(x) - P_{n}(x)$}
\sol{
	(a)
	\begin{center}
		\[ f(x) = e^{x}\] 
		\[ f'(x) = e^{x}\] 
		\[ f''(x) = e^{x}\] 
		\[ f^{(2)}(x) = e^{x}\] 
		\[ f^{(3)}(x) = e^{x}\] 
		\[ f^{(4)}(x) = e^{x}\] 
		\[ f^{(5)}(x) = e^{x}\] 
		\[ P_{5}(x) = e^{0} + e^{0}(x) + \frac{e^{0}}{2}(x)^{2} + \frac{e^{0}}{6}(x)^{3} + \frac{e^{0}}{24}(x)^{4} + \frac{e^{0}}{120}(x)^{5}\] 
		\[ P_{5}(x) = 1 + (x) + \frac{1}{2}(x)^{2} + \frac{1}{6}(x)^{3} + \frac{1}{24}(x)^{4} + \frac{1}{120}(x)^{5}\] 
	\end{center}
	(b) 
	\begin{center}
		\[ P_{n}(x) = \sum^{n}_{k=0} \frac{1}{k!}(x)^{k}\] 
	\end{center}
	(c) 
	\begin{center}
		\[ |f(x) - P_{n}(x)| \leq \frac{M(x-a)^{n+1}}{(n+1)}\] 
		\[ |f(x) - P_{n}(x)| \leq \frac{e^{x}(x)^{n+1}}{(n+1)!}, \text{where } |e^{z}| \leq M \text{ for all } z \text{ between 0 and } x \] 
	\end{center}
	(d) 
	\begin{center}
		
	\end{center}
}

\end{document}
